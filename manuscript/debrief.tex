\documentclass[10pt,letterpaper]{article}
\usepackage[top=0.85in,left=2.75in,footskip=0.75in,marginparwidth=2in]{geometry}

% use Unicode characters - try changing the option if you run into troubles with special characters (e.g. umlauts)
\usepackage[utf8]{inputenc}

% clean citations
\usepackage{cite}

% hyperref makes references clicky. use \url{www.example.com} or \href{www.example.com}{description} to add a clicky url
\usepackage{nameref,hyperref}

% line numbers
\usepackage[right]{lineno}

% improves typesetting in LaTeX
\usepackage{microtype}
\DisableLigatures[f]{encoding = *, family = * }

% text layout - change as needed
\raggedright
\setlength{\parindent}{0.5cm}
\textwidth 5.25in 
\textheight 8.75in

% Remove % for double line spacing
%\usepackage{setspace} 
%\doublespacing

% use adjustwidth environment to exceed text width (see examples in text)
\usepackage{changepage}

% adjust caption style
\usepackage[aboveskip=1pt,labelfont=bf,labelsep=period,singlelinecheck=off]{caption}

% remove brackets from references
\makeatletter
\renewcommand{\@biblabel}[1]{\quad#1.}
\makeatother

% headrule, footrule and page numbers
\usepackage{lastpage,fancyhdr,graphicx}
\usepackage{epstopdf}
\pagestyle{myheadings}
\pagestyle{fancy}
\fancyhf{}
\rfoot{\thepage/\pageref{LastPage}}
\renewcommand{\footrule}{\hrule height 2pt \vspace{2mm}}
\fancyheadoffset[L]{2.25in}
\fancyfootoffset[L]{2.25in}

% use \textcolor{color}{text} for colored text (e.g. highlight to-do areas)
\usepackage{color}

% define custom colors (this one is for figure captions)
\definecolor{Gray}{gray}{.25}

% this is required to include graphics
\usepackage{graphicx}

% use if you want to put caption to the side of the figure - see example in text
\usepackage{sidecap}

% use for have text wrap around figures
\usepackage{wrapfig}
\usepackage[pscoord]{eso-pic}
\usepackage[fulladjust]{marginnote}
\reversemarginpar

% Adding multirow.
\usepackage{multirow}

% Other required things:
\usepackage{color}
\usepackage{subcaption}
\captionsetup[subfigure]{justification=centering}
\newcommand{\code}[1]{\texttt{#1}}

\newcommand{\supptabdataset}{1}
\newcommand{\suppfignegative}{1}
\newcommand{\suppfigsimdesign}{2}
\newcommand{\suppfigsimresults}{3-7}
\newcommand{\suppfigrealresults}{8-12}

% document begins here
\begin{document}
\vspace*{0.35in}

% title goes here:
\begin{flushleft}
{\Large
    \textbf\newline{Distinguishing cells from empty droplets in droplet-based single-cell RNA sequencing data}
}
\newline

% authors go here:
%\\
Aaron T. L. Lun\textsuperscript{1,*},
Samantha Riesenfeld\textsuperscript{2,*},
Tallulah Andrews\textsuperscript{3,*},
The Phuong Dao\textsuperscript{4,*},
Tomas Gomes\textsuperscript{3,*}
and others
\\
\bigskip
\bf{1} Cancer Research UK Cambridge Institute, University of Cambridge, Li Ka Shing Centre, Robinson Way, Cambridge CB2 0RE, United Kingdom \\
\bf{2} Something... Broad? \\
\bf{3} Wellcome Trust Sanger Institute, Wellcome Genome Campus, Hinxton, Cambridge CB10 1SA, United Kingdom \\
\bf{4} Something... Columbia?
\\
\bigskip
* These authors contributed equally to this work.

\end{flushleft}

\section*{Introduction}
Recent advances in droplet-based protocols have revolutionized the field of single-cell transcriptomics by allowing tens of thousands of cells to be profiled in a single assay \cite{macosko2015highly,klein2015droplet,zheng2017massively}.
In these technologies, individual cells are captured into aqueous droplets in a water-in-oil emulsion.
Each droplet also contains a co-captured bead with primers for reverse transcription, where all primers on a single bead contain a cell barcode that is (effectively) unique to that bead.
The droplets serve as isolated reaction chambers in which cell lysis and reverse transcription are performed to obtain barcoded cDNA.
This is followed by breaking of the emulsion, amplification of the cDNA and construction of a sequencing library.
After sequencing, debarcoding is performed based on the cell barcode observed in each read sequence.
This yields an expression profile for each cell, typically in the form of unique molecular identifier (UMI) counts \cite{islam2014quantitative} for all annotated genes. 
The use of droplets increases throughput by at least an order of magnitude compared to protocols based on plates \cite{picelli2013smartseq2} or conventional microfluidics \cite{pollen2014low}, which is appealing for large-scale projects such as the Human Cell Atlas \cite{regev2017human}.

That said, the complexity of the sequencing data from droplet-based technologies poses a number of interesting challenges for low-level data processing.
One such challenge is the identification and removal of cell barcodes corresponding to empty droplets.
An empty droplet does not contain a cell but will still contain ``ambient'' RNA \cite{macosko2015highly}, i.e., cell-free transcripts in the solution in which the cells are suspended.
Ambient RNA may be actively secreted by cells or released upon cell lysis, the latter of which is particularly likely given the stresses of dissociation.
The presence of ambient RNA means that many empty droplets will contain material for reverse transcription and library preparation, resulting in non-zero total UMI counts for the corresponding barcodes.
However, the resulting expression profiles do not originate from any single cell and need to be removed prior to further analysis to avoid misleading conclusions.

Existing methods for removing empty droplets assume that droplets containing genuine cells should have more RNA, resulting in larger total UMI counts for the corresponding barcodes.
Zheng \textit{et al.} \cite{zheng2017massively} remove all barcodes with total counts below 10\% of the 99\textsuperscript{th} percentile of the $Y$ largest total counts (where $Y$ is defined as the expected number of cells to be captured on the Chromium device).
Macosko \textit{et al.} \cite{macosko2015highly} set the threshold at the knee point in the cumulative fraction of reads with respect to increasing total count.
While simple, the use of a one-dimensional filter on the total UMI count is suboptimal as it may discard small cells with low RNA content.
Droplets containing small cells may not be easily distinguishable from large empty droplets in terms of the total number of transcripts.
This problem is exacerbated by variable capture and amplification efficiency across droplets, which further mixes the distributions of total counts between empty and non-empty droplets.
A simple threshold on the total count forces the researcher into a difficult choice between the loss of small cells or an increase to the number of artifactual ``cells'' composed of ambient RNA.

In this report, we propose a new method for detecting empty droplets in droplet-based single-cell RNA sequencing (scRNA-seq) data.
We construct a profile of the ambient pool of RNA, and test each barcode for deviations from this profile using a Poisson-based model for the count distribution.
Barcodes with significant deviations are considered to be genuine cells, thus allowing recovery of cells with low total RNA content and small total UMI counts.
We combine our approach with an inflection point filter to ensure that barcodes with large total counts are always retained.
Using a variety of simulations, we demonstrate that our method outperforms any simple threshold on the total UMI count.
We also apply our method to several real data sets where we are able to recover more cells from both existing and new cell types.

\section*{Description of the method}

\subsection*{Testing for deviations from the ambient profile}
To construct the profile for the ambient RNA pool, we consider a threshold $T$ on the total UMI count.
The set $\mathcal{G}$ of all barcodes with total counts less than or equal to $T$ are considered to represent empty droplets.
The exact choice of $T$ does not matter, as long as (i) it is small enough so that droplets with genuine cells do not have total counts below $T$,
and (ii) there are sufficient counts to obtain a precise estimate of the ambient profile.
We set $T=100$ by default in our approach, motivated by examination of several real datasets (Supplementary Materials, Supplementary Figure~\suppfignegative{}).
We stress that $T$ is not the same as the threshold used in existing methods as barcodes with total counts greater than $T$ are not automatically considered to be cell-containing droplets.

The ambient profile is constructed by summing counts for each gene across $\mathcal{G}$.
Let $y_{gb}$ be the count for gene $g$ in barcode $b$.
We define the ambient count as 
\[
    A_{g} = \sum_{b \in \mathcal{G}} y_{gb} \;,
\]
yielding a count vector $\mathbf{A} = (A_1, \dots, A_N)$ for all $N$ genes.
(We assume that any gene with counts of zero for all barcodes has already been filtered out, as this provides no information for distiguishing between barcodes.)
We apply the Good-Turing algorithm to $\mathbf{A}$ to obtain the posterior expectation $\tilde{p}_g$ of the proportion of counts assigned to $g$ \cite{gale1995good}, using the \code{goodTuringProportions} function in the \textsf{edgeR} package \cite{robinson2010edgeR}.
This ensures that genes with zero counts in the ambient pool have non-zero proportions, avoiding the possibility of obtaining likelihoods of zero in downstream calculations.

Our null hypothesis is that free-floating transcripts in solution are randomly sampled into the empty droplets.
Each transcript molecule for gene $g$ is independently sampled with probability equal to the proportion $\tilde{p}_g$.
If we condition on the total count $t_b$ for each barcode $b$, we can model the counts for each barcode with a multinomial distribution.
We define the likelihood of obtaining the counts for barcode $b$ as 
\[
    L_b = t_b! \prod_{g=1}^N \frac{\tilde{p}_g}{y_{gb}!} \;.
\]
We use a Monte Carlo approach to compute the $p$-value for $b$.
We generate a new set of counts by randomly sampling from a multinomial distribution with probabilities set to $\tilde{p}_g$ for all $g$ and size equal to $t_b$.
We use the above formula to calculate the likelihood for this set of counts (denoted $L'_{bi}$, for iteration $i$), and we repeat this process for $R$ iterations. 
We use the method of Phipson and Smyth \cite{phipson2010permutation} to define the $p$-value as 
\[
    P_b = \frac{R_b +1 }{R + 1} \;,
\]
where $R_b$ is the number of iterations in which $L'_{bi} \le L_b$.
This strategy avoids $p$-values of zero, which is important during multiple testing correction.
$R$ is set to 10000 by default, and increasing this will yield more precise $p$-values at the cost of speed.

\subsection*{Detecting the knee point in the log-totals}
The procedure described above will identify barcodes that have count profiles that are significantly different from the ambient pool of RNA.
This will be the case for most cell-containing droplets, as the ambient pool is formed from many (lysed) cells and is unlikely to be representative of any single cell.
However, it is possible for some cell-containing droplets to have ambient-like expression profiles.
This can occur if the cell population is highly homogeneous or if one cell subpopulation contributes disproportionately to the ambient pool, e.g., if it is more prone to lysis.
Sequencing errors in the cell barcodes may also bias the estimates of the ambient proportions, by misassigning counts from cell-containing droplets to barcodes with low UMI totals.
This may result in spurious similarities between cells and the estimated ambient profile.

If we apply our procedure directly, barcodes corresponding to ambient-like cell-containing droplets will be incorrectly filtered out.
To avoid this, we combine our procedure with a conventional threshold on the total UMI count.
We rank all barcodes in order of decreasing $t_b$, and consider the function $f(.)$ of $\log(t_b)$ with respect to increasing log-rank.
The first ``knee'' point in this function corresponds to a transition between a distinct subset of barcodes with large totals and the majority of barcodes with smaller totals.
This is defined as the log-rank that maximizes the curvature
\[
    \frac{|f''|}{(1 + f'^2)^{1.5}} \;,
\]
and represents the point at which $f(.)$ begins to drop rapidly, marking the start of the transition between large and small totals.
In practice, we obtain $f(.)$ by fitting a smooth spline to $\log(t_b)$ against the log-rank.
This avoids multiplication of errors during numerical differentiation, which could lead to unstable estimates of the knee point.

Our assumption is that all barcodes with large total counts must represent cell-containing droplets, regardless of whether its count profile resembles the ambient pool.
This is based on the expectation that the distribution of the sizes of empty droplets should be unimodal, with a monotonic decreasing probability density as $t_b$ increases past the mode.
A distinct peak of large totals would not be consistent with this expected distribution.
We define the upper threshold $U$ as the $t_b$ at the knee point and retain all barcodes with $t_b \ge U$, regardless of their $P_b$.
This guarantees recovery of any barcodes with large total counts that potentially represent cell-containing droplets. 
We use the knee point rather than the inflection point as the $t_b$ at the former is larger, providing a more conservative threshold that avoids retention of empty droplets.

We stress that, despite the use of a threshold on $t_b$, our approach is different from existing methods due to the testing procedure.
Barcodes with $t_b$ below the knee point can still be retained if the count profile is significantly different from the ambient pool.
This is not possible with existing methods that would simply discard these barcodes.
Users can also set $U$ manually if automatic detection of the knee point fails for complex $f(.)$.
Alternatively, this mechanism can be disabled completely in favour of detecting cells solely based on their $p$-values.
This is more statistically rigorous as it avoids the selection of an \textit{ad hoc} threshold, but may result in the failure to detect large cells.

\subsection*{Correcting for multiple testing across barcodes}
We correct for multiple testing by controlling the false discovery rate (FDR) using the Benjamini-Hochberg method \cite{benjamini1995controlling}.
Putative cells are defined as those that have significantly poor fits to the ambient model at a specified FDR threshold.
We set the FDR threshold to 1\% by default, meaning that the expected proportion of empty droplets in the set of retained barcodes is no greater than 1\%.
Note that we only perform the correction on the $p$-values for barcodes that have $t_b$ greater than $T$.
This reduces the severity of the correction given that barcodes with lower $t_b$ will always be discarded.
Similarly, all barcodes with $t_b \ge U$ have their $p$-values set to zero during correction, as these barcodes are considered to be known true positives.

\section*{Results}

\subsection*{Performance on simulated data}
We named our method ``EmptyDrops'' and tested it on simulated data involving cells with different RNA content (see Methods, Supplementary Figure~\suppfigsimdesign{}).
Each simulated dataset was generated from real droplet-based scRNA-seq data (Supplementary Table~\supptabdataset{}) 
and contained one group of large cells with high RNA content and large $t_b$;
one group of small cells with low RNA content and small $t_b$; 
and a set of empty droplets with counts sampled from an ambient pool of RNA.
We applied EmptyDrops at a FDR of 1\% to determine the recall for each group of cells and the FDR among the detected barcodes.
We also tested methods that retain all cells with total UMI counts above a threshold.
The threshold was defined as the total $U$ at the knee point, as described above;
or using the quantile-based approach \cite{zheng2017massively} in the CellRanger software from 10X Genomics.

Results for a simulation based on a real dataset containing peripheral blood mononuclear cells (PBMCs) are shown in Figure~\ref{fig:simpbmc}.
All methods consistently detected the large cells but recall for the group of small cells was much lower.
EmptyDrops achieved the highest recall for both groups of cells, and detected approximately 2-5-fold more small cells than the other methods.
EmptyDrops was also able to control the FDR below the specified threshold of 1\%.
We observed similar performance in simulations based on other real datasets (Supplementary Figures~\suppfigsimresults{}).
Improved detection of small cells with EmptyDrops was particularly pronounced in simulations based on the neuronal and cell line datasets,
with recall of 40-80\% compared to below 10\% with the other methods.
Modest loss of FDR control ($\approx$ 2\%) with EmptyDrops was observed in some simulation scenarios, which is fully attributable to incorrect identification of the knee point.
This motivates the choice to disable automatic retention of cells with $t_b \ge U$, as discussed above.
By comparison, the observed FDR for the same scenarios was much higher (over 10\%) with CellRanger, indicating that EmptyDrops is still superior.

\begin{figure}[btp]
    \begin{center}
        \begin{subfigure}{0.49\textwidth}
        \includegraphics[width=\textwidth,trim=0mm 10mm 0mm 19mm,clip]{../simulations/results-sim/pics/pbmc4k_G1.pdf}
        \caption{}
        \end{subfigure}
        \begin{subfigure}{0.49\textwidth}
        \includegraphics[width=\textwidth,trim=0mm 10mm 0mm 19mm,clip]{../simulations/results-sim/pics/pbmc4k_G2.pdf}
        \caption{}
    \end{subfigure} \\[0.1in]
        \begin{subfigure}{0.49\textwidth}
        \includegraphics[width=\textwidth,trim=0mm 10mm 0mm 19mm,clip]{../simulations/results-sim/pics/pbmc4k_FDR.pdf}
        \caption{}
        \end{subfigure}
        \begin{subfigure}{0.49\textwidth}
        \includegraphics[width=\textwidth,trim=0mm 19mm 0mm 10mm,clip]{../simulations/results-sim/pics/legend.pdf}
        \end{subfigure}
    \end{center}
\caption{Results for the simulations based on the PBMC dataset, using three different methods for detecting cell-containing droplets.    
Simulation scenarios are labelled as $G_1/G_2$ where $G_1$ and $G_2$ are the number of barcodes in the group of large and small cells, respectively.
The recall for each group is shown as a proportion of the group size (a, b), and the FDR is calculated as the proportion of detected droplets that are empty (c).
Each point represents the result of one simulation iteration, while the bar represents the mean across 10 iterations and the error bars represent the standard error of the mean.
The dotted line represents the nominal FDR threshold (1\%) for EmptyDrops.
}
\label{fig:simpbmc}
\end{figure}

The poor performance of the threshold-based methods for small cells is expected.
Barcodes corresponding to small cells with little RNA have similar total UMI counts as barcodes corresponding to large empty droplets with high levels of ambient RNA.
The total UMI count cannot distinguish between these two possibilities, resulting in either reduced recall or a high false positive rate.
In contrast, EmptyDrops uses the expression profile for each droplet to distinguish small cells from the ambient profile with greater power.
Another benefit of EmptyDrops is its statistical rigour, allowing users to increase the stringency of the output by using a lower FDR threshold.
The effect of adjusting the parameters in the quantile-based CellRanger approach is less interpretable.

\subsection*{Performance on real data}
To determine how EmptyDrops behaved on real data, we applied it to detect cells in the PBMC dataset at a FDR of 1\% (Figure~\ref{fig:realpbmc}).
EmptyDrops was able to identify an appropriate $U$ using the knee point from the smoothed spline (Figure~\ref{fig:realpbmc}a).
We were also able to detect significant barcodes as those with low probabilities under the null multinomial model (Figure~\ref{fig:realpbmc}b).
Comparison of EmptyDrops to CellRanger indicated that most barcodes were detected by both methods (Figure~\ref{fig:realpbmc}c).
Barcodes that were only detected by EmptyDrops had low total counts (Figure~\ref{fig:realpbmc}d), consistent with our simulation results.
Conversely, CellRanger uniquely detected a number of cells with modest total counts.
This is attributable to the conservativeness of the knee point threshold, which ensures that empty droplets are not inadvertently retained.

\begin{figure}[btp]
    \begin{center}
    \begin{subfigure}{0.49\textwidth}
        \includegraphics[width=\textwidth,trim=0mm 0mm 0mm 20mm,clip]{../simulations/pics-real/rank_pbmc4k.pdf}
        \caption{}
    \end{subfigure}
    \begin{subfigure}{0.49\textwidth}
        \includegraphics[width=\textwidth,trim=0mm 0mm 0mm 20mm,clip]{../simulations/pics-real/dev_pbmc4k.pdf}
        \caption{}
    \end{subfigure}\\[0.1in]
    \begin{subfigure}[b]{0.49\textwidth}
        \includegraphics[width=\textwidth,trim=0mm 0mm 0mm 5mm,clip]{../simulations/pics-real/intersect_pbmc4k.pdf}
        \caption{}
    \end{subfigure}
    \begin{subfigure}[b]{0.49\textwidth}
        \includegraphics[width=\textwidth,trim=0mm 0mm 0mm 20mm,clip]{../simulations/pics-real/kept_pbmc4k.pdf}
        \caption{}
    \end{subfigure}
\end{center}
    \caption{Results of applying EmptyDrops and the other cell detection methods to the PBMC dataset.
        (a) A barcode rank plot showing the fitted spline used for knee point detection in EmptyDrops. 
        The detected inflection point is also shown.
        (b) The negative log-probability for each barcode in the multinomial model of EmptyDrops, plotted against the total count.
        Barcodes detected as putative cell-containing droplets at a FDR of 1\% are labelled in red.
        Only barcodes with $t_b > T$ are shown.
        (c) An UpSet plot \cite{lex2014upset} of the barcodes detected by each combination of methods (vertical bars).
        Horizontal bars represent the number of barcodes detected by each method.
        (d) Histogram outlines of the log-total count for barcodes detected by each method.
    }
\label{fig:realpbmc}
\end{figure}

We observed similar results in the other tested datasets (Supplementary Figures~\suppfigrealresults{}) where EmptyDrops consistently detected the greatest number of cells.
Increased retention of small cells with EmptyDrops was particularly pronounced in the neuronal datasets, where EmptyDrops was able to uniquely detect over a thousand cells.
Again, a small number of barcodes were uniquely detected by CellRanger in some datasets, though we note that this does not represent a fundamental deficiency of our method.
EmptyDrops can always be made to report a superset of the barcodes identified by CellRanger, simply by setting $U$ to the CellRanger-identified threshold.
This is not the default as the CellRanger threshold may not be appropriate -- moreover, it relies on knowledge of the expected number of cells, which may not be available or accurate.

To explore the differences between methods in more detail, we generated a $t$-stochastic neighbour embedding (t-SNE) plot \cite{van2008visualizing} of all barcodes that were detected by either method in the PBMC dataset.
We observed that the CellRanger-only barcodes clustered with barcodes that were detected by both methods (Figure~\ref{fig:realtsne}a).
This suggests that the conservativeness of EmptyDrops can be largely ignored, as it only results in the loss of some cells from a cluster that would have been detected anyway.
In contrast, the EmptyDrops-only barcodes formed a number of unique clusters.
We identified one of these clusters as containing platelets, based on the expression of GP9, PF4 and PPBP (Figure~\ref{fig:realtsne}b). 
This is not surprising as the total RNA content of a cell is often associated with its type/state.
The ability of EmptyDrops to retain small cells means that it can capture biology that would have been lost with CellRanger.

\begin{figure}[btp]
    \begin{subfigure}{0.49\textwidth}
        \includegraphics[width=\textwidth]{../analysis/pics/by_detection.pdf}
        \caption{}
    \end{subfigure}
    \begin{subfigure}{0.49\textwidth}
        \includegraphics[width=\textwidth]{../analysis/pics/by_platelet.pdf}
        \caption{}
    \end{subfigure} \\[0.05in]
    \begin{subfigure}{0.49\textwidth}
        \includegraphics[width=\textwidth]{../analysis/pics/by_ribo.pdf}
        \caption{}
    \end{subfigure}
    \begin{subfigure}{0.49\textwidth}
        \includegraphics[width=\textwidth]{../analysis/pics/by_mito.pdf}
        \caption{}
    \end{subfigure}
    \caption{$t$-SNE plots generated for the PBMC dataset.
        Each point represents a cell that is coloured by (a) detection status with either or both EmptyDrops and CellRanger, 
        or expression of (b) platelet genes GP9, PF4 and PPBP, (c) ribosomal protein genes or (d) mitochondrial genes.
        Expression in each cell was quantified as the sum of the normalized log-expression values across all genes in the relevant set.
    }
    \label{fig:realtsne}
\end{figure}

\section*{Discussion}
Droplet-based technologies are becoming increasingly popular for high-throughput single-cell transcriptomics.
However, little work has been performed to develop computational methods for distinguishing genuine cells from empty droplets.
Here, we describe EmptyDrops, a method to detect cell-containing barcodes based on significant deviation of the expression profiles from the pool of ambient RNA.
We use simulated data to demonstrate that EmptyDrops outperforms the strategy that is currently implemented in the CellRanger software suite.
Furthermore, EmptyDrops can recover biology in real 10X data that is lost using the CellRanger strategy.
Our results indicate that EmptyDrops is effective for cell detection in droplet-based scRNA-seq data.

A key assumption of our approach is that barcodes with very low UMI totals represent empty droplets.
This allows us to use these barcodes to obtain an estimate of the ambient pool.
However, this assumption may not be appropriate if the data set contains a subset of cells with very low RNA content.
In such cases, the estimate of the ambient expression profile will be biased, though this bias is likely to be small as relatively few transcripts will be contributed by cells with low RNA content.
Another potential source of bias may arise from sequencing errors in the cell barcode, such that transcripts from a cell-containing droplet are misassigned to an otherwise empty droplet.
This is mitigated to some extent by the use of designed cell barcodes in the GemCode protocol, which allow for error correction in the barcode \cite{zheng2017massively}.
However, it may be more of a problem in protocols where error correction of the barcodes is not possible \cite{macosko2015highly}.

% Note that the detection drops down to a dribble near total=100 in the real data.
% This is consistent with our assumptions.

An interesting question is whether the output of EmptyDrops is compatible with existing scRNA-seq analysis workflows.
In particular, many workflows recommend the removal of cells with low total numbers of expressed genes during quality control \cite{lun2016stepbystep,mccarthy2017scater}.
This would potentially result in the loss of cells that are recovered by EmptyDrops, defeating the purpose of using EmptyDrops in the first place.
Moreover, it remains to be seen whether downstream analyses can effectively model the very high frequency of zero counts in these small cells.
With improved recovery of small cells from droplet data, modifications may be required to existing scRNA-seq analysis pipelines.

\section*{Methods}

\subsection*{Obtaining the datasets}
All datasets were downloaded from the 10X Genomics website (\url{https://support.10xgenomics.com/single-cell-gene-expression/datasets}).
Only the ``raw'' count matrices were used to ensure that CellRanger filtering was not already applied to the cell barcodes.
Supplementary Table~\supptabdataset{} contains a brief summary of the datasets used in our study.

\subsection*{Evaluating performance with simulated data}
For a given real dataset, we computed the total sum of UMI counts $t_b$ for each barcode.
We identified the inflection point in the curve of $\log(t_b)$ against the log-rank using the \code{barcodeRanks} function from the \textsf{DropletUtils} package.
The set of all barcodes with $\log(t_b)$ below the inflection point was defined as the set of empty droplets $\mathcal{G}_0$.
Counts for all $b \in \mathcal{G}_0$ were summed together to create an ambient pool of RNA molecules.
(Here, we use the inflection point to be conservative with the definition of empty droplets.
This avoids the inclusion of cell-containing droplets in the ambient pool.)
To simulate known empty droplets, we constructed expression profiles for a new set of barcodes by sampling molecules from the ambient pool without replacement.
This was done such that the distribution of $t_b$ in our set was the same as that in $\mathcal{G}_0$. 
In this manner, we recapitulated the observed number of empty droplets and their total counts in our simulations.

To obtain $G_1$ large cells, we sampled from the set of barcodes with $\log(t_b)$ above the inflection point.
We used sampling with replacement to avoid problems in cases where $G_1$ is greater than the number of estimated cells in the dataset.
To generate $G_2$ small cells, we sampled from the same set of barcodes and downsampled the count vector for each barcode to 10\% of its original total,
using the \code{downsampleMatrix} function from the \textsf{DropletUtils} package.
This mimics the presence of small cells with low RNA content. 
We tested different simulation scenarios by setting $G_1$ or $G_2$ to 500 and 2000 cells.
The various components of the simulation are visualized in Supplementary Figure~\suppfigsimdesign{}.

We applied our EmptyDrops method to the simulated data at a FDR of 1\%. 
The recall was defined as the proportion of known cells from each group that were successfully detected.
The observed false discovery rate was defined as the proportion of detected barcodes that were known empty droplets.
We repeated this evaluation using the knee point approach, where all barcodes with total counts above the knee point were retained;
and with the CellRanger approach, implemented as described \cite{zheng2017massively} with the expected number of cells set to $G_1+G_2$ (i.e., the true number of simulated cells).

We generated simulated data based on each real dataset in Supplementary Table~\supptabdataset{}.
For each scenario and dataset, we repeated the simulation for 10 iterations.
We used each method in each iteration and collected performance metrics across all iterations.

\subsection*{Detecting cells in real data}
For each dataset in Supplementary Table~\supptabdataset{}, we applied EmptyDrops to detect cells at a FDR of 1\%. 
We also used the knee point approach, as well as the CellRanger approach where the expected number of cells was set to the reported value in Supplementary Table~\supptabdataset{}.
UpSet plots were created with using the \textsf{UpSetR} package \cite{lex2014upset}.

\subsection*{Analysis of real data}
We obtained one channel of the 10X PBMC data set from ???, and defined cells from the raw barcode counts using the EmptyDrops and CellRanger methods. 
For the CellRanger method, the expected number of cells was estimated as the number of cells identified at a FDR of 1\% using EmptyDrops (8524 cells).
This ensured a fair comparison between the two methods, though CellRanger was mostly robust to this parameter with 8318, 8381, and 8387 cells identified when the expected number of cells used was 5000, 8524, and 10000, respectively. 
For the set of barcodes detected by either method, we generated a t-SNE plot from the library-size normalized expression profiles using the Rtsne package (default parameters, perplexity of 30), using the top 500 highly variable genes. 
Library size normalization was performed by scaling the counts for each cell such that its total count was equal to the median total counts across all cells. 
Three groups enriched in cells identified only using EmptyDrops were segmented by eye, and marker genes were identified with a one-vs-all Wilcox-rank-sum test using a 5\% FDR threshold.

\bibliography{ref.bib}
\bibliographystyle{unsrt}

\end{document}
