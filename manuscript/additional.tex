\documentclass{article}
\usepackage{amsmath}
\usepackage[margin=3cm]{geometry}
\usepackage[hidelinks]{hyperref}

\makeatletter 
\renewcommand{\fnum@figure}{Supplementary \figurename~\thefigure}
\renewcommand{\fnum@table}{Supplementary \tablename~\thetable}
\makeatother

\usepackage[labelfont=bf]{caption}
\usepackage{subcaption}
\usepackage{graphicx}
\newcommand{\code}[1]{\texttt{#1}}

\usepackage{color}
\newcommand{\revised}[1]{\textcolor{red}{#1}}

\begin{document}

\begin{titlepage}
\vspace*{3cm}
\begin{center}


{\LARGE
Distinguishing cells from empty droplets in droplet-based single-cell RNA sequencing data
\par}

\vspace{0.75cm}

{\Large
    \textsc{Supplementary Materials}
\par
}
\vspace{0.75cm}

\large
by


\vspace{0.75cm}
Aaron T. L. Lun$^1$,
and others

\vspace{1cm}
\begin{minipage}{0.9\textwidth}
\begin{flushleft}
$^1$Cancer Research UK Cambridge Institute, University of Cambridge, Li Ka Shing Centre, Robinson Way, Cambridge CB2 0RE, United Kingdom \\[6pt]
$^2$Blah blah blah \\[6pt]
\end{flushleft}
\end{minipage}

\vspace{1.5cm}
{\large \today{}}

\vspace*{\fill}
\end{center}
\end{titlepage}

\begin{figure}
    \begin{center}
        \includegraphics[width=\textwidth]{../simulations/pics-sim/pbmc4k_2000_2000.pdf}
    \end{center}
\caption{Total count against the rank for each barcode in a simulation based on the PBMC dataset with $G_1=G_2=2000$.
Plots are shown for all barcodes, barcodes corresponding to empty droplets, and barcodes corresponding to large or small cells.
Ranks are calculated from the entire set of barcodes in all plots, for ease of comparison between plots.
All axes are on a log-scale.}
\end{figure}

\begin{table}
\caption{Summary of the datasets used to assess the various cell detection methods.
    All datasets were obtained from the 10X Genomics website.
The organism, cell type, estimated number of cells from CellRanger and the version of the CellRanger software used is shown for each dataset.}
\begin{center}
\begin{tabular}{l l l l r}
\hline
\textbf{Name} & \textbf{Organism} & \textbf{Cell type} & \textbf{Number} & \textbf{Version} \\
\hline
\code{293t}   & Human & 293T cell line & 2800 & 1.1.0 \\
\code{jurkat} & Human & Jurkat cell line & 3200 & 1.1.0 \\
\code{neuron\_9k} & Mouse & Brain cells & 9128 & 2.1.0 \\
\code{neurons\_900} & Mouse & Brain cells & 931 & 2.1.0 \\
\code{pbmc4k} & Human & PBMCs & 4340 & 2.1.0 \\
\code{t\_4k} & Human & Pan T cells & 4538 & 2.1.0 \\
\hline
\end{tabular}
\end{center}
\end{table}

\end{document}
