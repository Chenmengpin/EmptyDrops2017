\documentclass{article}
\usepackage{amsmath}
\usepackage[margin=3cm]{geometry}
\usepackage[hidelinks]{hyperref}

\makeatletter 
\renewcommand{\fnum@figure}{Supplementary \figurename~\thefigure}
\renewcommand{\fnum@table}{Supplementary \tablename~\thetable}
\makeatother

\usepackage[labelfont=bf]{caption}
\usepackage{subcaption}
\usepackage{graphicx}
\newcommand{\code}[1]{\texttt{#1}}

\usepackage{color}
\newcommand{\revised}[1]{\textcolor{red}{#1}}

\begin{document}

\begin{titlepage}
\vspace*{3cm}
\begin{center}


{\LARGE
Distinguishing cells from empty droplets in droplet-based single-cell RNA sequencing data
\par}

\vspace{0.75cm}

{\Large
    \textsc{Supplementary Materials}
\par
}
\vspace{0.75cm}

\large
by


\vspace{0.75cm}
Aaron T. L. Lun$^1$,
and others

\vspace{1cm}
\begin{minipage}{0.9\textwidth}
\begin{flushleft}
$^1$Cancer Research UK Cambridge Institute, University of Cambridge, Li Ka Shing Centre, Robinson Way, Cambridge CB2 0RE, United Kingdom \\[6pt]
$^2$Blah blah blah \\[6pt]
\end{flushleft}
\end{minipage}

\vspace{1.5cm}
{\large \today{}}

\vspace*{\fill}
\end{center}
\end{titlepage}

\section*{Motivating the choice of the total count threshold $T$}
The threshold $T$ should be chosen so that cell-containing droplets are not used to estimate the ambient profile.
Otherwise, our estimate will be distorted and we will have less power to discriminate between cells and empty droplets.
To check if this occurs in real data, we calculated a $p$-value against the ambient null hypothesis for each barcode $b \in \mathcal{G}$, 
i.e., with $t_b \le T$ where $T = 100$ by default.
Ideally, we should observe a uniform distribution of $p$-values under the assumption that all barcodes in $\mathcal{G}$ are genuinely empty.
However, if cell-containing droplets are present in $\mathcal{G}$, we should observe an enrichment of low $p$-values.
These low $p$-values either correspond to the cell-containing droplets themselves,
or to genuinely empty droplets that no longer fit to our distorted estimate of the ambient profile.

In most datasets, we observed a minor enrichment of low $p$-values in an otherwise uniform distribution (Supplementary Figure~\ref{fig:negative}). 
This demonstrates that our testing procedure mostly holds its size and suggests that distortions of the ambient profile due to cell-containing droplets are not a major issue when using $T=100$.
The exception is the \code{neurons\_900} dataset where we see a clear enrichment of very low and very large $p$-values.
Note that  this effect is not consistent with distortion of the ambient profile, which should only result in a skew towards low $p$-values.
Rather, we hypothesize that it is driven by violations of the independent sampling assumption (e.g., if transcript molecules are complexed and sampled together into droplets).
Positive correlations between molecules would increase the probability of obtaining droplets that are very similar or very different to the ambient profile.

\newpage
\begin{figure}[btp]
    \begin{center}
        \includegraphics[width=0.4\textwidth]{../simulations/pics-negcheck/hist_293t.pdf}
        \includegraphics[width=0.4\textwidth]{../simulations/pics-negcheck/hist_jurkat.pdf}
        \includegraphics[width=0.4\textwidth]{../simulations/pics-negcheck/hist_neuron_9k.pdf}
        \includegraphics[width=0.4\textwidth]{../simulations/pics-negcheck/hist_neurons_900.pdf}
        \includegraphics[width=0.4\textwidth]{../simulations/pics-negcheck/hist_pbmc4k.pdf}
        \includegraphics[width=0.4\textwidth]{../simulations/pics-negcheck/hist_t_4k.pdf}
    \end{center}
    \caption{Histograms of $p$-values for all barcodes with total UMI counts less than or equal to the threshold $T=100$.
        Each plot corresponds to a dataset in Supplementary Table~\ref{tab:datasets} where the $p$-value represents the deviation from the ambient profile.
    }
    \label{fig:negative}
\end{figure}

\begin{figure}[btp]
    \begin{center}
        \includegraphics[width=\textwidth]{../simulations/pics-sim/pbmc4k_2000_2000.pdf}
    \end{center}
\caption{Total count against the rank for each barcode in a simulation based on the PBMC dataset with $G_1=G_2=2000$.
Plots are shown for all barcodes, barcodes corresponding to empty droplets, and barcodes corresponding to large or small cells.
Ranks are calculated from the entire set of barcodes in all plots, for ease of comparison between plots.
All axes are on a log-scale.}
\end{figure}

\begin{table}[btp]
\caption{Summary of the datasets used to assess the various cell detection methods.
    All datasets were obtained from the 10X Genomics website.
The organism, cell type, estimated number of cells from CellRanger and the version of the CellRanger software used is shown for each dataset.}
\begin{center}
\begin{tabular}{l l l l r}
\hline
\textbf{Name} & \textbf{Organism} & \textbf{Cell type} & \textbf{Number} & \textbf{Version} \\
\hline
\code{293t}   & Human & 293T cell line & 2800 & 1.1.0 \\
\code{jurkat} & Human & Jurkat cell line & 3200 & 1.1.0 \\
\code{neuron\_9k} & Mouse & Brain cells & 9128 & 2.1.0 \\
\code{neurons\_900} & Mouse & Brain cells & 931 & 2.1.0 \\
\code{pbmc4k} & Human & PBMCs & 4340 & 2.1.0 \\
\code{t\_4k} & Human & Pan T cells & 4538 & 2.1.0 \\
\hline
\end{tabular}
\end{center}
\label{tab:datasets}
\end{table}

\end{document}
